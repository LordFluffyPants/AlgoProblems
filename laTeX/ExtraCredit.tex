\documentclass[a4paper,11pt, twocolumn]{article}


\title{Lashback Against Automation}
\author{John Wesley McDonald Hayhurst}

\begin{document}

\maketitle

\begin{abstract} 
While automate this takes a positive approach to the automation of our world, many people are opposed to algorithms taking over the world. This shortsighted view of algorithms hurts humanity as a whole. 
\end{abstract} 
\section{The Backlash of the General Public} 
It is no surprise that the general public is concerned that with algorithms able to do their job faster, better, and do not need to be paid. We can see this already with the rise of middle class jobs being taken over by machines. In fact a recent study has shown that more than nearly fifty percent of American jobs can be taken over by a machine.\footnote{http://www.theatlantic.com what jobs will robots take} Many humans especially ones within these middle class jobs are against automation. They are afraid that when their job is taken by robots that there will be nothing left form them to do. This is a very shortsighted view of algorithms and can hurt progress for humanity. 

\section{Humans need not apply} 
From humble beginnings as cave men, humans have grown and adapted to become more and more lazy. We created tools to help craft and shape the world around us. One of the biggest way we have shaped the world is through agriculture. Back when humans had to fend for themselves most if not everyone was responsible for providing food for themselves. Almost all Humans used to have to hunt to eat or die of starvation. Flash forward a couple thousand years or so and the closest thing a human does to hunt for their food is look down the isles of a grocery store. This number is not just a radical difference between days of old and modern times. In the last couple of years the percent of the population that makes the world's food supply has greatly decreased.\footnote{http://data.worldbank.org/indicator/SL.AGR.EMPL.ZS} Being able to say that you did not starve this week because your food did not out run you is pretty nice. What really allowed humans to be able to produce this much food was algorithms and machines. Algorithms helped create the market place for more efficient delivery methods to keep food fresh and not spoiled. Algorithms helped create more efficient machines to produce, grow, and plant crops across the world. This is only one aspect of your day to day life that algorithms have helped create or improve. 
 
Lets take a look at how machines and algorithms have already helped out a species in improving their quality of their life. Lets take a look at horses and the rise of the automobile. As technology progressed the demand to travel farther distances increased and horses were no longer cutting it. So with cars being so much better horses, horses became "unemployable". Yet horses today still have jobs. They have jobs as show horses, race horses, and even jobs we could not think of horses having before cars. Their are some horses that are used for therapy in the U.K.\footnote{http://www.equine-therapy-programs.com/behavioral.html} Now look at the quality of life of horses. They no longer have to commute humans around as their primary role in life. They are no longer doing agricultural work as machines have replaced them. Their primary role in life is significantly easier and more beneficial to the horse. So now imagine that humans the dominate species are considered "unemployable". Humans will still have jobs but they may not be the jobs that we think of. With what many might consider menial work left to the algorithms and machines we as humans are able to more freely create and express ourselves.\footnote{http://www.barcinno.com/human-creativity-will-thrive-robotic-world/} Not to mention that these algorithms cannot create themselves yet. Humans must create and maintain these algorithims that are stealing human jobs. Even now we see more and more college bound students entering STEM fields.\footnote{http://www.usnews.com/news/articles/2015/01/27/more-students-earning-degrees-in-stem-fields-report-shows}  
 
There are some economic implications for algorithms taking over human jobs as well. Consider the basic economic principle, supply and demand. This is the driving force for how all products are bought and sold. So how can this be applied to algorithms? If humans are put out of work due to algorithms and do not have enough money to afford the products that these algorithms create it creates a vicious cycle. The demand for algorithms to take over human jobs was created by an increase demand in those products. Then if the demand for those products go down because too many algorithms are taking over the jobs of humans then the demand for those algorithms decrease as well. Eventually from the standpoint of supply and demand we will reach an equilibrium of needing algorithms and being pushed out of our jobs because of them. 
 
\section{Ethical Implications} 
Their are a few ethical implications that we can derive from this argument. Is it ethical to create algorithms that can and will put people out of jobs? Is it ethical to promote your field of work when other fields of work are going to suffer because of that? Is it ethical under the standards of the IEEE code of ethics\footnote{http://www.ieee.org/about/corporate/governance/p7-8.html}? These are ethical implications that programmers must consider when designing such algorithms. For the question of is it ethical under the IEEE the answer is up for debate but the code does not violate any ethical principles. The problem comes when what is considered public interest. Is it in the public interest for people to keep their jobs now or is it in the public interest for some people to lose their jobs now to make the quality of life better in the future? Is it ethical to promote something that will promote your field of work will damaging other fields of work? In my personal opinion the answer is yes. As long as when you are creating these algorithms the purpose it not to eliminate that field of work but rather improve upon it. By making the field of work more efficient creating these algorithms are benefiting that field of work rather than hindering it. 
 
\section{Conclusion} 
In the end the backlash against creating algorithms that take away human jobs is misunderstood. While their might be some consequences here in the present, improvements to quality of life will outweigh any hardships. It is said that a society during its golden age has more freedom to express their creativity and create art. It can be shown that creating algorithms which will free up humans to be more creative could lead humanity to a new golden age. 
 

\end{document}
